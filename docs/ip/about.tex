\documentclass[]{article}
\usepackage{lmodern}
\usepackage{amssymb,amsmath}
\usepackage{ifxetex,ifluatex}
\usepackage{fixltx2e} % provides \textsubscript
\ifnum 0\ifxetex 1\fi\ifluatex 1\fi=0 % if pdftex
  \usepackage[T1]{fontenc}
  \usepackage[utf8]{inputenc}
\else % if luatex or xelatex
  \ifxetex
    \usepackage{mathspec}
  \else
    \usepackage{fontspec}
  \fi
  \defaultfontfeatures{Ligatures=TeX,Scale=MatchLowercase}
\fi
% use upquote if available, for straight quotes in verbatim environments
\IfFileExists{upquote.sty}{\usepackage{upquote}}{}
% use microtype if available
\IfFileExists{microtype.sty}{%
\usepackage{microtype}
\UseMicrotypeSet[protrusion]{basicmath} % disable protrusion for tt fonts
}{}
\usepackage[margin=1in]{geometry}
\usepackage{hyperref}
\hypersetup{unicode=true,
            pdftitle={About this Workshop},
            pdfborder={0 0 0},
            breaklinks=true}
\urlstyle{same}  % don't use monospace font for urls
\usepackage{graphicx,grffile}
\makeatletter
\def\maxwidth{\ifdim\Gin@nat@width>\linewidth\linewidth\else\Gin@nat@width\fi}
\def\maxheight{\ifdim\Gin@nat@height>\textheight\textheight\else\Gin@nat@height\fi}
\makeatother
% Scale images if necessary, so that they will not overflow the page
% margins by default, and it is still possible to overwrite the defaults
% using explicit options in \includegraphics[width, height, ...]{}
\setkeys{Gin}{width=\maxwidth,height=\maxheight,keepaspectratio}
\IfFileExists{parskip.sty}{%
\usepackage{parskip}
}{% else
\setlength{\parindent}{0pt}
\setlength{\parskip}{6pt plus 2pt minus 1pt}
}
\setlength{\emergencystretch}{3em}  % prevent overfull lines
\providecommand{\tightlist}{%
  \setlength{\itemsep}{0pt}\setlength{\parskip}{0pt}}
\setcounter{secnumdepth}{0}
% Redefines (sub)paragraphs to behave more like sections
\ifx\paragraph\undefined\else
\let\oldparagraph\paragraph
\renewcommand{\paragraph}[1]{\oldparagraph{#1}\mbox{}}
\fi
\ifx\subparagraph\undefined\else
\let\oldsubparagraph\subparagraph
\renewcommand{\subparagraph}[1]{\oldsubparagraph{#1}\mbox{}}
\fi

%%% Use protect on footnotes to avoid problems with footnotes in titles
\let\rmarkdownfootnote\footnote%
\def\footnote{\protect\rmarkdownfootnote}

%%% Change title format to be more compact
\usepackage{titling}

% Create subtitle command for use in maketitle
\providecommand{\subtitle}[1]{
  \posttitle{
    \begin{center}\large#1\end{center}
    }
}

\setlength{\droptitle}{-2em}

  \title{About this Workshop}
    \pretitle{\vspace{\droptitle}\centering\huge}
  \posttitle{\par}
    \author{}
    \preauthor{}\postauthor{}
    \date{}
    \predate{}\postdate{}
  

\begin{document}
\maketitle

\href{https://onlinelearningconsortium.org/olc-accelerate-2019-session-page/?session=7732}{Link
to session page.}

\hypertarget{brief-abstract}{%
\section{Brief Abstract}\label{brief-abstract}}

Are you an impostor? Full of doubt, inadequacy? Do you think your
success is just luck? These feelings lead to a destructive mindset of
stress, hesitancy, and disengagement. Together, we'll learn how to
recognize and address the impostor phenomenon and how to flip the script
on your own ``impostor'' dialogues.

\hypertarget{extended-abstract}{%
\section{Extended Abstract}\label{extended-abstract}}

While attending OLC Innovate 2018, a graduate student standing at the
back of the room noted that she didn't belong at the conference. It was
not because of the conference topics or structure. It was because she
felt like she was somehow not as knowledgeable as or not as experienced
as other attendees. This student, working on her doctorate in online
learning belonged at the conference, and yet she felt her experiences
and accomplishments in the online learning landscape were somehow not
adequate to even be in the same room with other attendees.

Philosopher Bertrand Russell wrote: ``The whole problem with the world
is that fools and fanatics are always so certain of themselves, and
wiser people so full of doubts.'' This doubt that Russell speaks of can
sometimes be positive, offering an opportunity to learn, grow, and
develop. But sometimes this doubt can be destructive, suggesting to our
innermost selves that we aren't good enough, smart enough, or able to
accomplish what is in front of us. This is likely what the graduate
student was feeling.

This type of ``self-doubt'' is sometimes referred to as the Impostor
Phenomenon, which some research estimates almost 70\% of successful
people have experienced (Gravoy, 2007). Impostor phenomenon (IP) is a
``psychological pattern. It is based on intense, secret feelings of
fraudulence in the face of success and achievement. If you suffer from
the impostor phenomenon, you believe that you don't deserve your
success; you're a phony who has somehow `gotten away with it.'' (Harvey
\& Katz, 1984, p.~3). Similarly, Tabaka (2018) described impostor
phenomenon as, ``When in the throes of an Imposter Syndrome struggle,
you may feel that you're the only person in your circle (or in the whole
world) who suffers from this level of self-doubt. In those moments,
you're certain that every label you've assigned to yourself, including
inadequate, incompetent, undeserving, unqualified, fake, and unequivocal
failure is absolutely accurate. The pain associated with the Imposter
Syndrome is very real, but the self-assessment that put you there is
not'' (p.~1).

As online leaders, designers, faculty, and support professionals, the
impostor phenomenon can be a destructive force, one that can stymie our
thinking in ways that shortchange any accomplishments, knowledge, or
experiences that got us to the point we are at today. As online
teachers, experts, leaders, designers, or support professionals,
addressing feelings in the impostor phenomenon are critical to our
professional performance (Cozarelli \& Major, 1990).

The purpose of this workshop is to start a conversation about the
impostor phenomenon feelings that online teaching and learning
professionals may experience and discuss strategies to address those
feelings. In this honest, personal, informative, and engaged express
workshop, participants will explore the impostor phenomenon as it
applies to their roles as online teachers, leaders, designers, and
support professionals and how they might recover a sense of confidence
in the work they do. Through a series of individual and small group
activities, including the wall of confidence, writing positive mantras,
and the values exercise, participants will be engaged in reflecting on
the impostor phenomenon and creating a plan for flipping the impostor
script.

By participating in this session, you will\ldots{}

\begin{itemize}
\tightlist
\item
  Self-assess your own impostor phenomenon level
\item
  Identify strategies for rewriting your own ``impostor'' dialogue.
\item
  Identify at least one person you will actively build a network with to
  further support and mentor one another within the structure and format
  that best supports you professional and academic growth.
\end{itemize}

\hypertarget{references}{%
\section{References}\label{references}}

Cozzarelli, C., \& Major, B. (1990). Exploring the validity of the
impostor phenomenon. Journal of Social and Clinical Psychology, 9(4),
401-417.\\
Gravois, John. (2007). You're Not Fooling Anyone. The Chronicle of
Higher Education, 54(11), A1 A32-A32.\\
Harvey, J.C. \& Katz, C. (1984). If I'm So Successful, Why do I Feel
Like A Fake? Random House: New York.\\
Tabaka, M. (2018). Here's What Famous High Achievers Are Doing to
Conquer Symptoms of the Imposter Syndrome. Retrieved from
\url{https://www.inc.com/marla-tabaka/famous-high-achievers-are-finally-talki}\ldots{}


\end{document}
